% Options for packages loaded elsewhere
\PassOptionsToPackage{unicode}{hyperref}
\PassOptionsToPackage{hyphens}{url}
%
\documentclass[
]{article}
\usepackage{amsmath,amssymb}
\usepackage{iftex}
\ifPDFTeX
  \usepackage[T1]{fontenc}
  \usepackage[utf8]{inputenc}
  \usepackage{textcomp} % provide euro and other symbols
\else % if luatex or xetex
  \usepackage{unicode-math} % this also loads fontspec
  \defaultfontfeatures{Scale=MatchLowercase}
  \defaultfontfeatures[\rmfamily]{Ligatures=TeX,Scale=1}
\fi
\usepackage{lmodern}
\ifPDFTeX\else
  % xetex/luatex font selection
\fi
% Use upquote if available, for straight quotes in verbatim environments
\IfFileExists{upquote.sty}{\usepackage{upquote}}{}
\IfFileExists{microtype.sty}{% use microtype if available
  \usepackage[]{microtype}
  \UseMicrotypeSet[protrusion]{basicmath} % disable protrusion for tt fonts
}{}
\makeatletter
\@ifundefined{KOMAClassName}{% if non-KOMA class
  \IfFileExists{parskip.sty}{%
    \usepackage{parskip}
  }{% else
    \setlength{\parindent}{0pt}
    \setlength{\parskip}{6pt plus 2pt minus 1pt}}
}{% if KOMA class
  \KOMAoptions{parskip=half}}
\makeatother
\usepackage{xcolor}
\usepackage[margin=1in]{geometry}
\usepackage{color}
\usepackage{fancyvrb}
\newcommand{\VerbBar}{|}
\newcommand{\VERB}{\Verb[commandchars=\\\{\}]}
\DefineVerbatimEnvironment{Highlighting}{Verbatim}{commandchars=\\\{\}}
% Add ',fontsize=\small' for more characters per line
\usepackage{framed}
\definecolor{shadecolor}{RGB}{248,248,248}
\newenvironment{Shaded}{\begin{snugshade}}{\end{snugshade}}
\newcommand{\AlertTok}[1]{\textcolor[rgb]{0.94,0.16,0.16}{#1}}
\newcommand{\AnnotationTok}[1]{\textcolor[rgb]{0.56,0.35,0.01}{\textbf{\textit{#1}}}}
\newcommand{\AttributeTok}[1]{\textcolor[rgb]{0.13,0.29,0.53}{#1}}
\newcommand{\BaseNTok}[1]{\textcolor[rgb]{0.00,0.00,0.81}{#1}}
\newcommand{\BuiltInTok}[1]{#1}
\newcommand{\CharTok}[1]{\textcolor[rgb]{0.31,0.60,0.02}{#1}}
\newcommand{\CommentTok}[1]{\textcolor[rgb]{0.56,0.35,0.01}{\textit{#1}}}
\newcommand{\CommentVarTok}[1]{\textcolor[rgb]{0.56,0.35,0.01}{\textbf{\textit{#1}}}}
\newcommand{\ConstantTok}[1]{\textcolor[rgb]{0.56,0.35,0.01}{#1}}
\newcommand{\ControlFlowTok}[1]{\textcolor[rgb]{0.13,0.29,0.53}{\textbf{#1}}}
\newcommand{\DataTypeTok}[1]{\textcolor[rgb]{0.13,0.29,0.53}{#1}}
\newcommand{\DecValTok}[1]{\textcolor[rgb]{0.00,0.00,0.81}{#1}}
\newcommand{\DocumentationTok}[1]{\textcolor[rgb]{0.56,0.35,0.01}{\textbf{\textit{#1}}}}
\newcommand{\ErrorTok}[1]{\textcolor[rgb]{0.64,0.00,0.00}{\textbf{#1}}}
\newcommand{\ExtensionTok}[1]{#1}
\newcommand{\FloatTok}[1]{\textcolor[rgb]{0.00,0.00,0.81}{#1}}
\newcommand{\FunctionTok}[1]{\textcolor[rgb]{0.13,0.29,0.53}{\textbf{#1}}}
\newcommand{\ImportTok}[1]{#1}
\newcommand{\InformationTok}[1]{\textcolor[rgb]{0.56,0.35,0.01}{\textbf{\textit{#1}}}}
\newcommand{\KeywordTok}[1]{\textcolor[rgb]{0.13,0.29,0.53}{\textbf{#1}}}
\newcommand{\NormalTok}[1]{#1}
\newcommand{\OperatorTok}[1]{\textcolor[rgb]{0.81,0.36,0.00}{\textbf{#1}}}
\newcommand{\OtherTok}[1]{\textcolor[rgb]{0.56,0.35,0.01}{#1}}
\newcommand{\PreprocessorTok}[1]{\textcolor[rgb]{0.56,0.35,0.01}{\textit{#1}}}
\newcommand{\RegionMarkerTok}[1]{#1}
\newcommand{\SpecialCharTok}[1]{\textcolor[rgb]{0.81,0.36,0.00}{\textbf{#1}}}
\newcommand{\SpecialStringTok}[1]{\textcolor[rgb]{0.31,0.60,0.02}{#1}}
\newcommand{\StringTok}[1]{\textcolor[rgb]{0.31,0.60,0.02}{#1}}
\newcommand{\VariableTok}[1]{\textcolor[rgb]{0.00,0.00,0.00}{#1}}
\newcommand{\VerbatimStringTok}[1]{\textcolor[rgb]{0.31,0.60,0.02}{#1}}
\newcommand{\WarningTok}[1]{\textcolor[rgb]{0.56,0.35,0.01}{\textbf{\textit{#1}}}}
\usepackage{graphicx}
\makeatletter
\def\maxwidth{\ifdim\Gin@nat@width>\linewidth\linewidth\else\Gin@nat@width\fi}
\def\maxheight{\ifdim\Gin@nat@height>\textheight\textheight\else\Gin@nat@height\fi}
\makeatother
% Scale images if necessary, so that they will not overflow the page
% margins by default, and it is still possible to overwrite the defaults
% using explicit options in \includegraphics[width, height, ...]{}
\setkeys{Gin}{width=\maxwidth,height=\maxheight,keepaspectratio}
% Set default figure placement to htbp
\makeatletter
\def\fps@figure{htbp}
\makeatother
\setlength{\emergencystretch}{3em} % prevent overfull lines
\providecommand{\tightlist}{%
  \setlength{\itemsep}{0pt}\setlength{\parskip}{0pt}}
\setcounter{secnumdepth}{5}
\setlength{\parindent}{1em} \usepackage{float} \renewcommand{\thesubsection}{Question (\alph{subsection})}
\usepackage{booktabs}
\usepackage{longtable}
\usepackage{array}
\usepackage{multirow}
\usepackage{wrapfig}
\usepackage{float}
\usepackage{colortbl}
\usepackage{pdflscape}
\usepackage{tabu}
\usepackage{threeparttable}
\usepackage{threeparttablex}
\usepackage[normalem]{ulem}
\usepackage{makecell}
\usepackage{xcolor}
\ifLuaTeX
  \usepackage{selnolig}  % disable illegal ligatures
\fi
\IfFileExists{bookmark.sty}{\usepackage{bookmark}}{\usepackage{hyperref}}
\IfFileExists{xurl.sty}{\usepackage{xurl}}{} % add URL line breaks if available
\urlstyle{same}
\hypersetup{
  pdftitle={EDS241: Assignment 2},
  pdfauthor={Melissa Widas},
  hidelinks,
  pdfcreator={LaTeX via pandoc}}

\title{EDS241: Assignment 2}
\author{Melissa Widas}
\date{02/03/2024}

\begin{document}
\maketitle

\textbf{Reminders:} Make sure to read through the setup in markdown.
Remember to fully report/interpret your results and estimates (in
writing) + present them in tables/plots.

\hypertarget{part-1-treatment-ignorability-assumption-and-applying-matching-estimators-19-points}{%
\section{Part 1 Treatment Ignorability Assumption and Applying Matching
Estimators (19
points):}\label{part-1-treatment-ignorability-assumption-and-applying-matching-estimators-19-points}}

The goal is to estimate the causal effect of maternal smoking during
pregnancy on infant birth weight using the treatment ignorability
assumptions. The data are taken from the National Natality Detail Files,
and the extract ``SMOKING\_EDS241.csv''' is a random sample of all
births in Pennsylvania during 1989-1991. Each observation is a
mother-infant pair. The key variables are:

\textbf{The outcome and treatment variables are:}

\indent birthwgt=birth weight of infant in grams

\indent tobacco=indicator for maternal smoking

\textbf{The control variables are:}

\indent mage (mother's age), meduc (mother's education), mblack (=1 if
mother identifies as Black), alcohol (=1 if consumed alcohol during
pregnancy), first (=1 if first child), diabete (=1 if mother diabetic),
anemia (=1 if mother anemic)

\begin{Shaded}
\begin{Highlighting}[]
\CommentTok{\# Load data for Part 1}
\NormalTok{smoking}\OtherTok{\textless{}{-}} \FunctionTok{read\_csv}\NormalTok{(here}\SpecialCharTok{::}\FunctionTok{here}\NormalTok{(}\StringTok{"Assignment{-}2"}\NormalTok{, }\StringTok{"data"}\NormalTok{, }\StringTok{"SMOKING\_EDS241.csv"}\NormalTok{))}
\end{Highlighting}
\end{Shaded}

\hypertarget{mean-differences-assumptions-and-covariates-3-pts}{%
\subsection{\texorpdfstring{Mean Differences, Assumptions, and
Covariates \emph{(3
pts)}}{Mean Differences, Assumptions, and Covariates (3 pts)}}\label{mean-differences-assumptions-and-covariates-3-pts}}

\begin{enumerate}
\def\labelenumi{\alph{enumi})}
\tightlist
\item
\end{enumerate}

\begin{itemize}
\tightlist
\item
  What is the mean difference in birth weight of infants with smoking
  and non-smoking mothers {[}1 pts{]}?
\item
  Under what assumption does this correspond to the average treatment
  effect of maternal smoking during pregnancy on infant birth weight
  {[}0.5 pts{]}?
\item
  Calculate and create a table demonstrating the differences in the mean
  proportions/values of covariates observed in smokers and non-smokers
  (remember to report whether differences are statistically significant)
  and discuss whether this provides empirical evidence for or against
  this assumption. Remember that this is observational data.
\item
  What other quantitative empirical evidence or test could help you
  assess the former assumption? {[}1.5 pts: 0.5 pts table, 1 pts
  discussion{]}
\end{itemize}

\begin{Shaded}
\begin{Highlighting}[]
\DocumentationTok{\#\# Calculate mean difference. Remember to calculate a measure of statistical significance}

\NormalTok{smoking\_t }\OtherTok{\textless{}{-}} \FunctionTok{t.test}\NormalTok{(smoking}\SpecialCharTok{$}\NormalTok{birthwgt[smoking}\SpecialCharTok{$}\NormalTok{tobacco}\SpecialCharTok{==}\DecValTok{0}\NormalTok{], smoking}\SpecialCharTok{$}\NormalTok{birthwgt[smoking}\SpecialCharTok{$}\NormalTok{tobacco}\SpecialCharTok{==}\DecValTok{1}\NormalTok{])}

\FunctionTok{diff}\NormalTok{(smoking\_t}\SpecialCharTok{$}\NormalTok{estimate)}
\end{Highlighting}
\end{Shaded}

\begin{verbatim}
## mean of y 
## -244.5394
\end{verbatim}

\begin{itemize}
\item
  On average babies born to mothers who smoke are 244.53 grams less than
  babies born to mothers who do not smoke in Pennsylvania between 1989
  and 1991 (p-value of less than 1 out of 1000 making this measure
  statistically significant).
\item
  This corresponds to the average treatment effect of maternal smoking
  during pregnancy on infant birth weight when the assumption that the
  treatment which is smoking, is not correlated with any other variables
  or factors.
\end{itemize}

\begin{Shaded}
\begin{Highlighting}[]
\CommentTok{\# Selecting binary and continuous variables from the dataset}
\NormalTok{pretreat\_binary }\OtherTok{\textless{}{-}}\NormalTok{ smoking }\SpecialCharTok{\%\textgreater{}\%}
  \FunctionTok{select}\NormalTok{(anemia, diabete, tobacco, alcohol, mblack)}
\NormalTok{pretreat\_continuous }\OtherTok{\textless{}{-}}\NormalTok{ smoking }\SpecialCharTok{\%\textgreater{}\%}
  \FunctionTok{select}\NormalTok{(mage, meduc, birthwgt, tobacco)}
\CommentTok{\# Initialize empty data frames to store results of tests}
\NormalTok{prop\_test\_results }\OtherTok{\textless{}{-}} \FunctionTok{data.frame}\NormalTok{()}
\NormalTok{t\_test\_results }\OtherTok{\textless{}{-}} \FunctionTok{data.frame}\NormalTok{()}
\CommentTok{\# Identifying binary variables for proportion tests}
\NormalTok{binary\_vars }\OtherTok{\textless{}{-}} \FunctionTok{names}\NormalTok{(pretreat\_binary)}

\ControlFlowTok{for}\NormalTok{ (var }\ControlFlowTok{in}\NormalTok{ binary\_vars) \{}
\CommentTok{\# Splitting the data into treated and untreated groups for the current variable}
\NormalTok{treated }\OtherTok{\textless{}{-}}\NormalTok{ pretreat\_binary }\SpecialCharTok{\%\textgreater{}\%} \FunctionTok{filter}\NormalTok{(tobacco }\SpecialCharTok{==} \DecValTok{1}\NormalTok{) }\SpecialCharTok{\%\textgreater{}\%} \FunctionTok{pull}\NormalTok{(}\SpecialCharTok{!!}\FunctionTok{sym}\NormalTok{(var))}
\NormalTok{untreated }\OtherTok{\textless{}{-}}\NormalTok{ pretreat\_binary }\SpecialCharTok{\%\textgreater{}\%} \FunctionTok{filter}\NormalTok{(tobacco }\SpecialCharTok{==} \DecValTok{0}\NormalTok{) }\SpecialCharTok{\%\textgreater{}\%} \FunctionTok{pull}\NormalTok{(}\SpecialCharTok{!!}\FunctionTok{sym}\NormalTok{(var))}
\CommentTok{\# Performing the proportion test}
\NormalTok{prop\_test\_result }\OtherTok{\textless{}{-}} \FunctionTok{prop.test}\NormalTok{(}\AttributeTok{x =} \FunctionTok{c}\NormalTok{(}\FunctionTok{sum}\NormalTok{(treated), }\FunctionTok{sum}\NormalTok{(untreated)),}
\AttributeTok{n =} \FunctionTok{c}\NormalTok{(}\FunctionTok{length}\NormalTok{(treated), }\FunctionTok{length}\NormalTok{(untreated)),}
\AttributeTok{correct =} \ConstantTok{FALSE}\NormalTok{)}
\CommentTok{\# Storing the tidy results of the proportion test in the data frame}
\NormalTok{prop\_test\_result\_tidy }\OtherTok{\textless{}{-}}\NormalTok{ broom}\SpecialCharTok{::}\FunctionTok{tidy}\NormalTok{(prop\_test\_result)}
\NormalTok{prop\_test\_result\_tidy}\SpecialCharTok{$}\NormalTok{Variable }\OtherTok{\textless{}{-}}\NormalTok{ var}
\NormalTok{prop\_test\_results }\OtherTok{\textless{}{-}} \FunctionTok{rbind}\NormalTok{(prop\_test\_results, prop\_test\_result\_tidy)}
\NormalTok{\}}

\CommentTok{\# Identifying continuous variables for t{-}tests}
\NormalTok{continuous\_vars }\OtherTok{\textless{}{-}} \FunctionTok{names}\NormalTok{(pretreat\_continuous)}

\CommentTok{\# for (var in continuous\_vars) \{}
\CommentTok{\# \# Dynamically creating the formula for the t{-}test}
\CommentTok{\# formula \textless{}{-} as.formula(paste(var, "\textasciitilde{} tobacco"))}
\CommentTok{\# \# Performing the t{-}test}
\CommentTok{\# t\_test\_result \textless{}{-} t.test(formula, data = pretreat\_continuous)}
\CommentTok{\# \# Storing the tidy results of the t{-}test in the data frame}
\CommentTok{\# t\_test\_result\_tidy \textless{}{-} broom::tidy(t\_test\_result)}
\CommentTok{\# t\_test\_result\_tidy$Variable \textless{}{-} var}
\CommentTok{\# t\_test\_results \textless{}{-} rbind(t\_test\_results, t\_test\_result\_tidy)}
\CommentTok{\# \}}


\NormalTok{mage }\OtherTok{\textless{}{-}} \FunctionTok{t.test}\NormalTok{(mage }\SpecialCharTok{\textasciitilde{}}\NormalTok{ tobacco, }\AttributeTok{data =}\NormalTok{ pretreat\_continuous)}
\NormalTok{meduc }\OtherTok{\textless{}{-}} \FunctionTok{t.test}\NormalTok{(meduc }\SpecialCharTok{\textasciitilde{}}\NormalTok{ tobacco, }\AttributeTok{data =}\NormalTok{ pretreat\_continuous)}
\NormalTok{birthwgt }\OtherTok{\textless{}{-}} \FunctionTok{t.test}\NormalTok{(birthwgt }\SpecialCharTok{\textasciitilde{}}\NormalTok{ tobacco, }\AttributeTok{data =}\NormalTok{ pretreat\_continuous)}

\NormalTok{t\_test\_result\_tidy }\OtherTok{\textless{}{-}}\NormalTok{ broom}\SpecialCharTok{::}\FunctionTok{tidy}\NormalTok{(mage, meduc, birthwgt)}
\NormalTok{t\_test\_result\_tidy}\SpecialCharTok{$}\NormalTok{Variable }\OtherTok{\textless{}{-}}\NormalTok{ var}
\NormalTok{t\_test\_results }\OtherTok{\textless{}{-}} \FunctionTok{rbind}\NormalTok{(t\_test\_results, t\_test\_result\_tidy)}

\CommentTok{\# Combining the results of proportion and t{-}tests into a single data frame}
\NormalTok{combined\_results }\OtherTok{\textless{}{-}} \FunctionTok{bind\_rows}\NormalTok{(}
\NormalTok{prop\_test\_results }\SpecialCharTok{\%\textgreater{}\%}
  \FunctionTok{select}\NormalTok{(Variable, estimate1, estimate2, p.value),}
\NormalTok{t\_test\_results }\SpecialCharTok{\%\textgreater{}\%}
  \FunctionTok{select}\NormalTok{(Variable, estimate1, estimate2, p.value)}
\NormalTok{)}

\CommentTok{\# Creating a table for output using kable and kableExtra}
\NormalTok{combined\_results\_table }\OtherTok{\textless{}{-}} \FunctionTok{kable}\NormalTok{(combined\_results, }\AttributeTok{format =} \StringTok{"latex"}\NormalTok{,}
\AttributeTok{col.names =} \FunctionTok{c}\NormalTok{(}\StringTok{"Variable"}\NormalTok{, }\StringTok{"Proportion or Mean Treated"}\NormalTok{,}
\StringTok{"Proportion or Mean Control"}\NormalTok{, }\StringTok{"P{-}Value"}\NormalTok{),}
\AttributeTok{caption =} \StringTok{"Treated and Untreated Pre{-}treatment Proportion and T{-} Test"}\NormalTok{) }\SpecialCharTok{|\textgreater{}} 
\FunctionTok{kable\_styling}\NormalTok{(}\AttributeTok{font\_size =} \DecValTok{7}\NormalTok{, }\AttributeTok{latex\_options =} \StringTok{"hold\_position"}\NormalTok{)}

\CommentTok{\# Displaying the table}
\NormalTok{combined\_results\_table}
\end{Highlighting}
\end{Shaded}

\begin{table}[!h]
\centering
\caption{\label{tab:unnamed-chunk-3}Treated and Untreated Pre-treatment Proportion and T- Test}
\centering
\fontsize{7}{9}\selectfont
\begin{tabular}[t]{l|r|r|r}
\hline
Variable & Proportion or Mean Treated & Proportion or Mean Control & P-Value\\
\hline
anemia & 0.0141031 & 0.0078005 & 0.0000000\\
\hline
diabete & 0.0175187 & 0.0173636 & 0.8858005\\
\hline
tobacco & 1.0000000 & 0.0000000 & 0.0000000\\
\hline
alcohol & 0.0441825 & 0.0071033 & 0.0000000\\
\hline
mblack & 0.1354121 & 0.1086279 & 0.0000000\\
\hline
mblack & 27.4530853 & 25.5385632 & 0.0000000\\
\hline
\end{tabular}
\end{table}

\hypertarget{ate-and-covariate-balance-3-pts}{%
\subsection{\texorpdfstring{ATE and Covariate Balance \emph{(3
pts)}}{ATE and Covariate Balance (3 pts)}}\label{ate-and-covariate-balance-3-pts}}

\begin{enumerate}
\def\labelenumi{\alph{enumi})}
\setcounter{enumi}{1}
\tightlist
\item
  Assume that maternal smoking is randomly assigned conditional on the
  observable covariates listed above. Estimate the effect of maternal
  smoking on birth weight using an OLS regression with NO linear
  controls for the covariates {[}0.5 pts{]}. Perform the same estimate
  including the control variables {[}0.5 pts{]}. Next, compute indices
  of covariate imbalance between the treated and non-treated regarding
  these covariates (see example file from class). Present your results
  in a table {[}1 pts{]}. What do you find and what does it say
  regarding whether the assumption you mentioned responding to a) is
  fulfilled? {[}1 pts{]}
\end{enumerate}

\begin{Shaded}
\begin{Highlighting}[]
\CommentTok{\# ATE Regression univariate}


\CommentTok{\# ATE with covariates}


\CommentTok{\# Present Regression Results}


\CommentTok{\# Covariate balance}


\CommentTok{\# Balance Table }
\end{Highlighting}
\end{Shaded}

\hypertarget{propensity-score-estimation-3-pts}{%
\subsection{\texorpdfstring{Propensity Score Estimation \emph{(3
pts)}}{Propensity Score Estimation (3 pts)}}\label{propensity-score-estimation-3-pts}}

\begin{enumerate}
\def\labelenumi{\alph{enumi})}
\setcounter{enumi}{2}
\tightlist
\item
  Next, estimate propensity scores (i.e.~probability of being treated)
  for the sample, using the provided covariates. Create a regression
  table reporting the results of the regression and discuss what the
  covariate coefficients indicate and interpret one coefficient {[}1.5
  pts{]}. Create histograms of the propensity scores comparing the
  distributions of propensity scores for smokers (`treated') and
  non-smokers (`control'), discuss the overlap and what it means {[}1.5
  pts{]}.
\end{enumerate}

\begin{Shaded}
\begin{Highlighting}[]
\DocumentationTok{\#\# Propensity Scores}


\DocumentationTok{\#\# PS Histogram Unmatched }
\end{Highlighting}
\end{Shaded}

\hypertarget{matching-balance-3-pts}{%
\subsection{\texorpdfstring{Matching Balance \emph{(3
pts)}}{Matching Balance (3 pts)}}\label{matching-balance-3-pts}}

\begin{enumerate}
\def\labelenumi{(\alph{enumi})}
\setcounter{enumi}{3}
\tightlist
\item
  Next, match treated/control mothers using your estimated propensity
  scores and nearest neighbor matching. Compare the balancing of
  pretreatment characteristics (covariates) between treated and
  non-treated units in the original dataset (from c) with the matched
  dataset (think about comparing histograms/regressions) {[}2 pts{]}.
  Make sure to report and discuss the balance statistics {[}1 pts{]}.
\end{enumerate}

\begin{Shaded}
\begin{Highlighting}[]
\DocumentationTok{\#\# Nearest{-}neighbor Matching}

\DocumentationTok{\#\# Covariate Imbalance post matching: }


\DocumentationTok{\#\# Histogram of PS after matching}
\end{Highlighting}
\end{Shaded}

\hypertarget{ate-with-nearest-neighbor-3-pts}{%
\subsection{\texorpdfstring{ATE with Nearest Neighbor \emph{(3
pts)}}{ATE with Nearest Neighbor (3 pts)}}\label{ate-with-nearest-neighbor-3-pts}}

\begin{enumerate}
\def\labelenumi{(\alph{enumi})}
\setcounter{enumi}{4}
\tightlist
\item
  Estimate the ATT using the matched dataset. Report and interpret your
  result (Note: no standard error or significance test is required here)
\end{enumerate}

\begin{Shaded}
\begin{Highlighting}[]
\DocumentationTok{\#\# Nearest Neighbor }

\DocumentationTok{\#\# ATT}
\end{Highlighting}
\end{Shaded}

\hypertarget{ate-with-wls-matching-3-pts}{%
\subsection{\texorpdfstring{ATE with WLS Matching \emph{(3
pts)}}{ATE with WLS Matching (3 pts)}}\label{ate-with-wls-matching-3-pts}}

\begin{enumerate}
\def\labelenumi{\alph{enumi})}
\setcounter{enumi}{5}
\tightlist
\item
  Last, use the original dataset and perform the weighted least squares
  estimation of the ATE using the propensity scores (including
  controls). Report and interpret your results, here include both size
  and precision of estimate in reporting and interpretation.
\end{enumerate}

\begin{Shaded}
\begin{Highlighting}[]
\DocumentationTok{\#\# Weighted least Squares (WLS) estimator Preparation}


\DocumentationTok{\#\# Weighted least Squares (WLS) Estimates}


\DocumentationTok{\#\# Present Results}
\end{Highlighting}
\end{Shaded}

\hypertarget{differences-in-estimates-1-pts}{%
\subsection{\texorpdfstring{Differences in Estimates \emph{(1
pts)}}{Differences in Estimates (1 pts)}}\label{differences-in-estimates-1-pts}}

\begin{enumerate}
\def\labelenumi{\alph{enumi})}
\setcounter{enumi}{6}
\tightlist
\item
  Explain why it was to be expected given your analysis above that there
  is a difference between your estimates in e) and f)?
\end{enumerate}

\newpage

\hypertarget{part-2-panel-model-and-fixed-effects-6-points}{%
\section{Part 2 Panel model and fixed effects (6
points)}\label{part-2-panel-model-and-fixed-effects-6-points}}

\indent We will use the progresa data from last time as well as a new
dataset. In the original dataset, treatment households had been
receiving the transfer for a year. Now, you get an additional dataset
with information on the same households from before the program was
implemented, establishing a baseline study (from 1997), and the same
data we worked with last time (from 1999). \indent *Note: You will need
to install the packages plm and dplyr (included in template preamble).
Again, you can find a description of the variables at the bottom of PDF
and HERE.

\hypertarget{estimating-effect-with-first-difference-3-pts-1.5-pts-estimate-1.5-pts-interpretation}{%
\subsection{\texorpdfstring{Estimating Effect with First Difference
\emph{(3 pts: 1.5 pts estimate, 1.5 pts
interpretation)}}{Estimating Effect with First Difference (3 pts: 1.5 pts estimate, 1.5 pts interpretation)}}\label{estimating-effect-with-first-difference-3-pts-1.5-pts-estimate-1.5-pts-interpretation}}

Setup: Load the new baseline data (progresa\_pre\_1997.csv) and the
follow-up data (progresa\_post\_1999.csv) into R. Note that we created a
time denoting variable (with the same name, `year') in BOTH datasets.
Then, create a panel dataset by appending the data (i.e.~binding the
dataset row-wise together creating a single dataset). We want to examine
the same outcome variable as before, value of animal holdings (vani).

\begin{Shaded}
\begin{Highlighting}[]
\FunctionTok{rm}\NormalTok{(}\AttributeTok{list=}\FunctionTok{ls}\NormalTok{()) }\CommentTok{\# clean environment}

\DocumentationTok{\#\# Load the datasets}
\NormalTok{pre\_1997 }\OtherTok{\textless{}{-}} \FunctionTok{read\_csv}\NormalTok{(here}\SpecialCharTok{::}\FunctionTok{here}\NormalTok{(}\StringTok{"Assignment{-}2"}\NormalTok{, }\StringTok{"data"}\NormalTok{, }\StringTok{"progresa\_pre\_1997.csv"}\NormalTok{))}
\NormalTok{post\_1999 }\OtherTok{\textless{}{-}} \FunctionTok{read\_csv}\NormalTok{(here}\SpecialCharTok{::}\FunctionTok{here}\NormalTok{(}\StringTok{"Assignment{-}2"}\NormalTok{, }\StringTok{"data"}\NormalTok{, }\StringTok{"progresa\_post\_1999.csv"}\NormalTok{))}

\DocumentationTok{\#\# Append post to pre dataset }
\CommentTok{\#progresa \textless{}{-} rbind(progresa\_pre\_1997, progresa\_post\_1999)}
\end{Highlighting}
\end{Shaded}

\begin{enumerate}
\def\labelenumi{\alph{enumi})}
\tightlist
\item
  Estimate a first-difference (FD) regression manually, interpret the
  results briefly (size of coefficient and precision!) \indent *Note:
  Calculate the difference between pre- and post- program outcomes for
  each family. To do that, follow these steps and the code given in the
  R-template:
\end{enumerate}

\begin{Shaded}
\begin{Highlighting}[]
\DocumentationTok{\#\#\# Code included to help get you started}
\DocumentationTok{\#\# i. Sort the panel data in the order in which you want to take differences, i.e. by household and time.}

\DocumentationTok{\#\# Create first differences of variables}
\CommentTok{\# progresa \textless{}{-} progresa \%\textgreater{}\% }
\CommentTok{\#   arrange(hhid, year) \%\textgreater{}\% }
\CommentTok{\#   group\_by(hhid)}

\DocumentationTok{\#\# ii. Calculate the first difference using the lag function from the dplyr package.}
\CommentTok{\#     mutate(vani\_fd = vani {-} dplyr::lag(vani)) }

\DocumentationTok{\#\# iii. Estimate manual first{-}difference regression (Estimate the regression using the newly created variables.)}
\CommentTok{\# fd\_manual \textless{}{-} lm(vani\_fd \textasciitilde{} ...)}
\end{Highlighting}
\end{Shaded}

\hypertarget{fixed-effects-estimates-2-pts-1-pts-estimate-1.5-interpretation}{%
\subsection{\texorpdfstring{Fixed Effects Estimates \emph{(2 pts: 1 pts
estimate, 1.5
interpretation)}}{Fixed Effects Estimates (2 pts: 1 pts estimate, 1.5 interpretation)}}\label{fixed-effects-estimates-2-pts-1-pts-estimate-1.5-interpretation}}

\begin{enumerate}
\def\labelenumi{\alph{enumi})}
\setcounter{enumi}{1}
\tightlist
\item
  Now also run a fixed effects (FE or `within') regression and compare
  the results. Interpret the estimated treatment effects briefly (size
  of coefficient and precision!)
\end{enumerate}

\begin{Shaded}
\begin{Highlighting}[]
\DocumentationTok{\#\# Fixed Effects Regression}

\DocumentationTok{\#\# Present Regression Results}
\end{Highlighting}
\end{Shaded}

\hypertarget{first-difference-and-fixed-effects-and-omitted-variable-problems-1-pts}{%
\subsection{\texorpdfstring{First Difference and Fixed Effects and
Omitted Variable Problems \emph{(1
pts)}}{First Difference and Fixed Effects and Omitted Variable Problems (1 pts)}}\label{first-difference-and-fixed-effects-and-omitted-variable-problems-1-pts}}

\begin{enumerate}
\def\labelenumi{\alph{enumi})}
\setcounter{enumi}{2}
\tightlist
\item
  Explain briefly how the FD and FE estimator solves a specific omitted
  variable problem? Look at the example on beer tax and traffic
  fatalities from class to start thinking about ommitted variables. Give
  an example of a potential omitted variable for the example we are
  working with here that might confound our results? For that omitted
  variable, is a FE or FD estimator better? One example is enough.
\end{enumerate}

\end{document}
